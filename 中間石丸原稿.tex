%%% 関西大学総合情報学部 松下研究室 中間発表用 TeX テンプレート Ver 1.2 (2015/8/12) %%%

% スタイルファイルは matsushita-zemi.cls を使います
\documentclass[a4j]{matsushita-zemi} 

%図表を使うための設定です
\usepackage[dvipdfmx]{graphicx}
\graphicspath{{./fig/}}
 \usepackage{ascmac}
 \usepackage{framed}

%% タイトル (長くなる場合は\\で適宜改行すること)
\title{コミックの指示的要約の基礎検討}


%%% 氏名 (姓・名の間は半角スペース) %%%
\idnum{情12-0033}
\author{石丸 亜美}


% --------------------------------------------------------
\begin{document}
\maketitle

%%% 以下,本文 %%%
\section{はじめに}
\label{background} 

近年、電子書籍の発展・普及に伴いPCやPDA、スマートフォンといったディジタル端末で読むことのできるコミックが急速に増加しつつある[1]。2014年度の電子書籍市場規模は1266億円で前年比の35%増、電子雑誌市場規模は145億円で前年比の39%増といずれも拡大した。さらに電子書籍市場の約8割をコミックが占めている。以降もコミック市場は拡大していくと予想される。従来の紙媒体のコミックとは異なり、ディジタル端末を媒体とするコミック(以下、ディジタルコミック)は、物理的な制約(e.g.,表示形式の固定,見開きが一つのまとまり)がないため、従来のコミックの枠にとらわれない様々な利用(e.g.,指定した箇所を検索する、コミックを要約する)が可能になる[2]。

コミックが電子化されることでコミックに情報(e.g.,相関関係,登場する場所の地図情報)を付与するといった紙媒体とは異なるコミックの新たな情報アクセス(紙媒体では目次とかページ数を表示させてるが、結局最初から読み返してしまう。その手間をなくすこと。)も可能になると期待される。本研究ではコミックコンテンツ内の各コマへのアクセス性の向上を目指している。アクセス性が向上することで、任意のシーンへのアクセスが容易になり、1度見たコミックの「あるシーン」を探したいといった要求にも応えることができる。各シーンへのアクセスを容易にするには何処にどういったシーンが描かれているのかを把握する必要がある。しかし現在の技術では自動でコミックのコマ内で何が起きているのかというコマの内容を認識することが難しい。また、計算機はコマ内の情報量の偏り(e.g.,あるコマは背景のみ、あるコマで突然新しい登場人物が出てくる)があるため話の一連の流れを理解することが容易ではない。加え、コミックはコマ内にあらゆる要素が混在しているマルチモーダルコンテンツである。従って目的を実現するためにコマ内の情報を減らし統一することが求められる。本稿では目的の実現に向けた基礎検討としてコマ内の情報を統一し内容理解が可能であるかの検証を試みる。

目的を達成するにあたり本研究では指示的要約手法に着目する。指示的要約とは必要な場合には原文を参照することを前提にして、原文を読むかどうかの判断をする指針となるための要約である[3]。指示的要約を用いることでユーザの求めるシーンへ柔軟にアクセスすることが可能になると期待される。これによりユーザは見たいシーンを素早く判断することができ、見たいシーンへのアクセスが容易になる。

本研究ではコミックのアクセス支援の実現を目指した基礎検討を行っていく。コミックからコマ内に書かれている情報を抜き出すために、作品中に含まれる構成要素を明確にする必要がある。そこでまず実際の作品のコマ内に含まれる構成要素の収集・分析・構造化を行い、データの抽出を行った。本稿では抽出したデータをもとにドラえもん1巻を対象としコマ内のセリフ情報に特化したタイムライン形式システムの実装を行った。


\section{論文の構成}
\label{works} 

本章では、コミックを対象とした指示的要約を行う上でコミックの要素の役割を把握する際に必要になるためコミックに関する定義について記述していく。

コミックの定義
 コミックは基本的に絵とテキストで構成されている。さらに絵と文字をコマという枠で囲み、それを連続させることで登場人物の動きや時間経過を表現している。夏目は、コミックには最低限コマという形式と絵という内容があり、現在のコミックとよばれているものの、ほとんどがコマは連続したものであると指摘している[3]。さらに夏目はこのコマと絵という構造に言葉が介在してくる、としている。また、コミックの表現として深層の部分(e.g.,時系列、シーン)と表層の部分(e.g.,コマ、吹き出し、オノマトペ)の2つに大きく分類される。

シーン、場面の定義(深層的な部分)
 コミックにおけるシーンとは複数の連続したコマによって成り立つ。シーンの句切れの具体例としては「風景のみが描画されたコマが入る(場所は変わったことを示す)」、「前のシーンでは登場しなかった登場人物が介入してくる」などといったものがあげられる。シーンの定義は様々で小方らの研究では、シーンは同一空間と時間的連続性によって定義できるとしている[4]。また田村らの研究では、時系列情報、場所情報の変化によって区切られた単位をシーンとしている[5]。さらにStamらは裂け目や中断がないと感じられる空間・時間的な継続性と定義している[6]。このようにシーンはコマ・オノマトペなどと違い明示されておらず明確な定義することは難しい。また、人は連続したコマの違いや一連の流れ(e.g., 登場人物のいないコマにある吹き出しの発話者(前に書かれているコマの発話者のセリフ))を見分けることができる。

コマ、ふきだし、オノマトペ、漫符、背景の定義(表層的な部分)
コマ
 コマは基本的に線によって区切られており、1コマ内に登場人物やセリフやオノマトペが描写されている。夏目は、コマの働きを3つあげている。
・	時間分節
法則に従い読んでいくことで、読者に時間経過の効果を与える
・	圧縮と解放
コマの大きさ、形、その変化によって生まれ、マンガの面白さをつくる効果を与えている
・	空間表象
絵の枠を限定し、絵の意味を背後で支える


ふきだし
 吹き出しは、発話者は誰なのか、登場人物がどんな感情でそのセリフを言っているのかを表現する。そのため吹き出しの形は、激しい口調で発言していればとげとげしくなり(3)、小さな声で発言している場合の吹き出しは小さかったり、点線で表現されたりする。吹き出しは、どの登場人物が発言したのかを示すために突起がついている。この突起がいくつかの泡状になっている吹き出しもあり、この場合は登場人物の思っていることなどを表している。中には突起がないものも存在するがナレーションである場合や登場人物の付近にセリフがある場合が多い。

オノマトペ
 オノマトペは擬音語・擬態語のことで映画やアニメにおける効果音と同様の機能を持っており、表現の拡張の役割を果たす。しかし、映画やアニメにおける効果音との違いは、登場人物が動いた際に鳴る音や、物が置かれたり壊れた際に鳴ったりする音(e.g,トントン、パタパタ)など現実世界でも聞こえるであろうとされる音の他に、様子や雰囲気など現実世界では聞こえない音(e.g,シーン、ガミガミ)を文字として描写している点である(4)。このオノマトペも、音の大きさやその場面の雰囲気に合わせて、文字の大きさや形が変化する。

漫符
 オノマトペと同様に人物の表情・感情に対して表現の拡張の役割を果たす。

背景



\begin{small}
  \begin{framed}
    \begin{enumerate}

    \item 序論 
      \begin{itemize}
      \item 目的と対象を明示する
      \end{itemize}

    \item 関連研究
      \begin{itemize}
      \item 適切な参考文献(査読付論文,国際会議原稿)を引用する
      \item 多少異なる論文も参考にする
      \item 既存の研究と自分の研究の比較表を作る
      \end{itemize}
      
    \item 提案/デザイン指針
      \begin{itemize}
      \item どのように問題を解決するか
      \item 問題解決の鍵となる考え方は何か
      \item その実現にはどのようなデザインが必要か
      \item どの点について新規性を認められるか
      \end{itemize}
      
    \item 実装
      \begin{itemize}
      \item どのようなプログラム構成か
      \item 他人が再現するのに足る情報提示ができているか
      \end{itemize}
      
    \item 実験
      \begin{itemize}
      \item 実験の統制はどのように行ったか
      \item どのようなデータが得られたか
      \end{itemize}
      
    \item 考察
      \begin{itemize}
      \item どのような分析手法を用いるか
      \item 仮説は検証されたか
      \end{itemize}
      
    \item 結論
      \begin{itemize}
      \item 何が明らかになったか
      \item 何が問題として残されたか
      \end{itemize}
      
    \item 謝辞

    \item 参考文献
    \end{enumerate}
  \end{framed}
\end{small}



\section{論文フォーマットの指針}
\label{sec:format}

以下では,論文フォーマットの指針について述べる.これに従って原稿を執筆
してほしい.なお,\LaTeX を用いた一般的な文章作成技術については,奥村
本\cite{okumura} 等を参考にされたい.


% ------------------------------------------------------------
\subsection{本文}
指摘が不十分な箇所があるかも知れないが,迷ったときは,情報処理学会の
投稿論文に準拠していただくか,先生に問い合わせていただきたい.また,
このファイルは適宜更新し,より合理的かつ理解しやすいフォーマットへと
進化させていきたいと考えているので,諸兄の協力を願う.


% ------------------------------------------------------------
\subsubsection{見出し}

見出しに用いるコマンドは \verb|\section|, \verb|\subsection|, \verb|\subsubsection|
の 3 種類である.原稿執筆の際は,不必要に階層を深くしないよう留意し,
中間発表の原稿では \verb|\subsubsection| の使用はなるべく控えること.

% ------------------------------------------------------------
\subsubsection{行送り}

このスタイルでは 2 段組を採用している.改行は,一行空行を設けることで行
い,原則として段落中では \verb|\\| による改行は行わないこと.ま
た,\verb|\vspace| や\verb|\vskip| を用いたスペースの調整は行わないこ
と.


% ------------------------------------------------------------
\subsubsection{フォントサイズ}

フォントサイズは,スタイルファイルによって自動的に設定されるため,基本
的には著者が自分でフォントサイズを変更する必要はない.なお本文のデフォ
ルトのフォントサイズは 10pt である.


% ------------------------------------------------------------
\subsubsection{句読点}

句点には全角の「.」を,読点には全角の「,」を各々用い,「。」や「、」
は使わないこと.ただし,英文中や数式中で「.」や「,」を使う場合には半角
文字を使うこと (e.g., $[x, y]=[3.1, 2.5]$).括弧と句読点の順には注意し
て欲しい(この文のように括弧の後に句点を打つ).

% ------------------------------------------------------------
\subsubsection{全角文字と半角文字}

全角文字と半角文字の両方にある文字は次のように使い分ける.

\begin{enumerate}
\item 括弧は全角の「(」と「)」を用いる.ただし,書誌データでは半角の
  「(」と「)」を用いる.

\item 英数字,空白,記号類は半角文字を用いる(句読点は除く).

\item カタカナは全角文字を用い,半角は何があっても使わないこと.

\item 引用符では開きと閉じを区別する.開きの引用符には \verb|``|(半角
  の \verb|`| をふたつ)を,閉じの引用符には \verb|''|(半角
  の \verb|'| をふたつ)を各々用いること.
\end{enumerate}

% ------------------------------------------------------------
\subsubsection{箇条書き}

箇条書きを行う場合は,目的に応じて \verb|enumerate| 環境(番号),
\verb|itemize| 環境(丸印),\verb|description| 環境(自分で定義)を各々
用いること.


% ------------------------------------------------------------
\subsubsection{脚注}

脚注は \verb|\footnote| コマンドを使って書くと,ページ単位に脚注が生成される\footnote{脚注の例.}.
なお,ページ内に複数の脚注がある場合,参照記号は \LaTeX を 2 回実行しない
と正しく反映されないので注意すること\footnote{二つ目の脚注.}.
 
また場合によっては,脚注をつけた位置と脚注本体とを別の段に置くほうが良い
こともある.この場合には,\verb|\footnotemark| コマンドや \verb|\footnotetext| コ
マンドを使って対処すること.

% ------------------------------------------------------------
\subsection{数式}\label{sec:Item}

%4.4.1
\subsubsection{本文中の数式}

本文中の数式は \verb|$| と \verb|$|, \verb|\(| と \verb|\)|, あるいは \verb|math| 環境のいず
れで囲んでもよい.

%4.4.2
\subsubsection{別組の数式}

別組数式 (displayed math) については \verb|$| と \verb|$| で囲むのでは
なく,\verb|\[| と \verb|\]| で囲むこと.数式に番号を付与したい場合
は,\verb|equation| 環境(数式が 1 行の場合),あるい
は \verb|eqnarray| 環境(数式が複数行の場合)のいずれかの環境を用いる.
以下に \verb|\[| と \verb|\]| で囲んだ例を示す.
\[
\mathit{div}_{G_i}(d_{jy}) 
\stackrel{\mathrm{def}}{=}\{d_{ix} \in \mathit{gr}(G_i) | d_{ix} \preceq d_{jy}\}
\]


%4.4.3
\subsubsection{eqnarray環境}

互いに関連する別組の数式が 2 行以上連続して現れる場合には,単に
\verb|\[| と \verb|\]|,あるい は \verb|\begin{equation}|
と \verb|\end{equation}| で囲った数式を書き並べるのではなく,
\verb|\begin|\allowbreak\verb|{eqnarray}| と \verb|\end{eqnarray}| 
を使って,等号(あるいは不等号)の位置で縦揃えを行なった方が読みやすい.
以下に \verb|eqnarray| 環境の例を示す.イコールの位置で揃っていることに注目.
%
\begin{eqnarray}
\Delta_l &=& \sum_{i=l|_1}^L\frac{\delta_{i}}{\Psi }\\
\frac{\partial y}{\partial x} &=& e^{-ax} \Bigl\{ \int e^{ax} Q(x)dx + C\Bigl\}
\end{eqnarray}
%

%4.5
\newpage
\subsection{図}

1段の幅におさまる図は,図\ref{fig:single} の形式で指定する.このとき,
位置の指定に \verb|[h]| オプションは使わない.

\begin{figure}[tb]
  \centering\includegraphics[clip, width=.95\columnwidth]{figureSample.eps}
  \caption{1段幅の図のサンプル}
  \label{fig:single}
\end{figure}


\begin{figure}[t]
  \begin{minipage}{0.49\hsize}
    \begin{center}
      \includegraphics[clip, width=.8\textwidth]{figureSample2.eps}\\
      \small{(a) 右図}
    \end{center}
  \end{minipage}%
  \begin{minipage}{0.49\hsize}
    \begin{center}
      \includegraphics[clip, width=.8\textwidth]{figureSample2.eps}\\
      \small{(a) 左図}
    \end{center}
  \end{minipage}
  \vspace{2pt}
  \caption{並べた図のサンプル}
  \label{fig:twoGraphs}
\end{figure}


\begin{figure*}[tb]
  \centering\includegraphics[clip, width=.9\textwidth]{figureSample3.eps}
  \caption{2段幅の図のサンプル}
  \label{fig:double}
\end{figure*}


またひとつの図のなかに複数の図を並べて表示したい場合には,\verb|minipage|
環境を使って,図\ref{fig:twoGraphs} のようにすることで実現できる.

2 段の幅にまたがる図は,図\ref{fig:double} の形式で指定する.
位置指定のオプションは \verb|[t]| しか使えない.

図の中身では本文と違い,どのような大きさのフォントを使用しても構わない
が,刷り上がり時に読めるサイズであることに留意すること.図の中身とし
て,encapsulate されたPostScriptファイル(いわゆるEPSファイル)を読み込
むことができる.mac\TeX の場合は読み込みには,プリアンブルで
%
\begin{framed}
    \verb|\usepackage[dvipdfmx]{graphicx}|
\end{framed}
%

\noindent
を行った上で,\verb|\includegraphics| コマンドを図を埋め込む箇所に置き,
その引数にファイル名(など)を指定する.Windows\TeX の場合
は,\verb|[dvipdfmx]| のところを\verb|[dviout]|というオプションに変更す
ること.

%4.6
\subsection{表}

表の罫線はなるべく少なくするのが,仕上がりをすっきりさせるコツである.
罫線をつける場合には,一番上の罫線には二重線を使い,左右の端には縦の罫
線をつけない(表\ref{tab:example}).また,見出し(\verb|\caption|)は,
表の場合は図と違って上側であるので注意していただきたい.

\begin{table}[tb] 
  \caption{表の例} 
  \label{tab:example}
  \hbox to\hsize{\hfil
    \begin{tabular}{l|lll}
      \hline\hline
           & column1     & column2  & column3 \\
      \hline
      row1 &	item 1,1 & item 2,1 & ---\\
      row2 &	---      & item 2,2 & item 3,2 \\
      row3 &	item 1,3 & item 2,3 & item 3,3 \\
      row4 &	item 1,4 & item 2,4 & item 3,4 \\
      \hline
    \end{tabular}\hfil}
\end{table}

%4.7
\subsection{参考文献・謝辞}

参考文献は \verb|.bib| ファイルに記述し,\BibTeX でコンパイルする方法を
推奨する.

%4.7.1
\subsubsection{参考文献の参照}

本文中で参考文献を参照する場合には \verb|\cite| を使用する.参照されたラ
ベルは自動的にソートされ番号が付与される.

\begin{framed}
文献 \verb|\cite{okumura, companion}| は \LaTeX の総合的な解説書である.
\end{framed}
\noindent
と書くと,
\begin{framed}
文献\cite{okumura, companion}は \LaTeX の総合的な解説書である.
\end{framed}
\noindent
が得られる.

%4.7.2
\subsubsection{参考文献リスト}

参考文献リストには,原則として本文中で引用した文献のみを列挙する.順序
は参照順あるいは第一著者の苗字のアルファベット順とする.文献リスト
は\BibTeX と\verb|matsort.bst|(アルファベット順)を用いて作
り,\verb|\bibliograhpystyle| と \verb |\bibliography| コマンドにより利
用することができる.これらを用いれば,規定の体裁にあったものができるの
で,できるだけ利用していただきたい.何らかの理由で thebibliography 環境
で文献リストを「手作り」しなければならない場合は,できるところま
で \BibTeX で作り,でき上がった \verb|.bbl| ファイルを加工すると工数が
減り効率的である(人工知能学会全国大会などオンライン予稿集の場合がそれ
に相当する).

\subsubsection{参考文献のタイプ}

詳細は,\verb|bibsample.bib| を参照されたい.卒業論文や修士論文で主に使
われるのは,書籍用の \verb|@Book|(e.g., \cite{okumura,companion}),原
著論文用の \verb|@article|(e.g., \cite{j024, j027}),国際会議予稿集用
の\verb|@inproceedings|(e.g., \cite{c028})である.なお,国内の全国大
会,研究会は基本的に国際会議と同様に\verb|@inproceedings|を用いること
(e.g., \cite{m139, m174}).なお,人工知能学会のように,オンラインプロ
シーディングスしかない場合はページの代わりに,発表番号
を \verb|presentation| のところに書く(e.g., \cite{m186}).詳細
は \verb|bibsample.bib|中のエントリ \verb|m186| を参照のこと.


%4.7.3
\subsubsection{謝辞}

謝辞がある場合には,参考文献リストの直前に,\verb|\section*{謝辞}| とい
う章を設けて記述する.科研費を始めとする外部予算を受けている場合は,必
ずここに研究費に対する謝辞「本研究は日本学術振興会科学研究費補助金基盤
研究(x)(課題番号:xxxxxx)の支援により実施された.」のように記述する
こと.


\section*{謝辞}
本テンプレートは情報処理学会の投稿論文フォーマットを参考にしている.記
して謝意を表す.


%%% 参考文献 %%%
\bibliographystyle{matsort}
\bibliography{bibsample.bib}

\end{document}